\documentclass[12pt, a4paper]{article}

% --- Γενικά Πακέτα ---
\usepackage[utf8]{inputenc} % Input encoding
\usepackage{fontspec}     % For font selection (recommended for XeLaTeX/LuaLaTeX)
\usepackage[greek,english]{babel} % Language support, AFTER fontspec can sometimes be better
\usepackage{graphicx} 
\usepackage{amsmath, amssymb, amsfonts} 
\usepackage[a4paper, margin=1in]{geometry} 
\usepackage{hyperref} 
\hypersetup{
    colorlinks=true,
    linkcolor=blue,
    filecolor=magenta,      
    urlcolor=cyan,
    pdftitle={scRNA-seq Analysis Application Report}, % Title in English for metadata
    pdfpagemode=FullScreen,
    pdfencoding=auto 
}
\usepackage{caption} 
\usepackage{listings} 
\usepackage{enumitem} 
\usepackage{booktabs} 
\usepackage{float} 

% --- Font Settings (with fontspec) ---
% Ensure the font is installed on your system if compiling locally with XeLaTeX/LuaLaTeX
\setmainfont{Times New Roman} % Main font
% \setsansfont{Arial} % Example if a separate sans-serif is needed
% \setmonofont{Courier New} % Example for monospaced font

% --- Language Settings ---
% Main language will be Greek
% \selectlanguage{greek} % This will be set after \begin{document}

% --- Listings Settings (Code) ---
\lstset{
    language=Python,
    basicstyle=\footnotesize\ttfamily, 
    keywordstyle=\color{blue},
    commentstyle=\color{green!70!black},
    stringstyle=\color{red},
    showstringspaces=false,
    tabsize=2,
    breaklines=true,
    breakatwhitespace=true,
    numbers=left,
    numberstyle=\tiny\color{gray},
    frame=single, 
    captionpos=b 
}

% --- Title, Authors, Date ---
\title{Ανάπτυξη Διαδραστικής Εφαρμογής για Ανάλυση Δεδομένων Μοριακής Βιολογίας (scRNA-seq)}
\author{Στεργιούλας Γεώργιος (inf2021216) \and Αναγνωστόπουλος Φίλιππος (inf2021014)} % Using \and for author separation
\date{Μάιος 2025}

% --- Document Start ---
\begin{document}
\selectlanguage{greek} % Set main language to Greek
\maketitle 

% --- Abstract ---
\begin{abstract}
\noindent 
Αυτή η εργασία παρουσιάζει το σχεδιασμό και την υλοποίηση μιας διαδραστικής εφαρμογής Streamlit για την ανάλυση δεδομένων single-cell RNA sequencing (scRNA-seq). 
Η εφαρμογή ενσωματώνει ένα τυπικό pipeline ανάλυσης scRNA-seq, επιτρέποντας στους χρήστες να ανεβάζουν τα δεδομένα τους, να εκτελούν βήματα όπως ποιοτικός έλεγχος, ενοποίηση δεδομένων, μείωση διαστατικότητας, ομαδοποίηση κυττάρων, ανάλυση διαφορικά εκφρασμένων γονιδίων και χαρακτηρισμό κυτταρικών τύπων. 
Η εφαρμογή παρέχει παραμετροποίηση των βασικών βημάτων και οπτικοποιήσεις των αποτελεσμάτων. Επιπλέον, η εφαρμογή έχει γίνει Dockerized για εύκολη ανάπτυξη και αναπαραγωγιμότητα.
\\ \\ 
\textbf{Λέξεις Κλειδιά:} scRNA-seq, Streamlit, Python, Ανάλυση Δεδομένων, Οπτικοποίηση, Docker.
\end{abstract}

\tableofcontents 
\newpage

% --- Εισαγωγή ---
\section{Εισαγωγή}
\label{sec:eisagogi}
Η ανάλυση δεδομένων single-cell RNA sequencing (scRNA-seq) έχει φέρει επανάσταση στην κατανόηση της ετερογένειας των κυτταρικών πληθυσμών και των μοριακών μηχανισμών που διέπουν βιολογικές διεργασίες. 
Ωστόσο, η πολυπλοκότητα των δεδομένων και των αναλυτικών pipelines απαιτεί εξειδικευμένες γνώσεις και υπολογιστικούς πόρους, καθιστώντας συχνά την πρόσβαση σε αυτές τις αναλύσεις δύσκολη για ερευνητές χωρίς ισχυρό βιοπληροφορικό υπόβαθρο.

Σκοπός του παρόντος project είναι η ανάπτυξη μιας φιλικής προς το χρήστη, διαδραστικής web εφαρμογής που απλοποιεί την εκτέλεση βασικών βημάτων ανάλυσης scRNA-seq. 
Η εφαρμογή, υλοποιημένη με τη χρήση Python και Streamlit, στοχεύει να παρέχει μια ολοκληρωμένη λύση από την εισαγωγή των δεδομένων έως την παραγωγή οπτικοποιήσεων και την εξαγωγή αποτελεσμάτων.

Σε αυτή την αναφορά, περιγράφεται λεπτομερώς ο σχεδιασμός, η υλοποίηση και η λειτουργικότητα της εφαρμογής. 
Αρχικά, παρουσιάζεται ο σχεδιασμός της υλοποίησης, ακολουθούμενος από τα UML διαγράμματα που αποτυπώνουν την αρχιτεκτονική. 
Στη συνέχεια, αναλύονται οι τεχνικές λεπτομέρειες της υλοποίησης κάθε συνιστώσας της εφαρμογής. 
Παρουσιάζονται παραδείγματα οπτικοποιήσεων που παράγει η εφαρμογή και, τέλος, περιγράφεται η διαδικασία Dockerization για τη διασφάλιση της αναπαραγωγιμότητας και της εύκολης διανομής.

% --- Σχεδιασμός της Υλοποίησης ---
\section{Σχεδιασμός της Υλοποίησης}
\label{sec:sxediasmos}
Ο σχεδιασμός της εφαρμογής βασίστηκε στην ανάγκη για μια αρθρωτή και επεκτάσιμη αρχιτεκτονική. 
Επιλέχθηκε η Python ως κύρια γλώσσα προγραμματισμού λόγω της πληθώρας βιβλιοθηκών για βιοπληροφορική ανάλυση (π.χ., Scanpy, AnnData, Pandas) και το Streamlit για την ταχεία ανάπτυξη διαδραστικών web διεπαφών.

Το pipeline ανάλυσης που ενσωματώνεται στην εφαρμογή περιλαμβάνει τα ακόλουθα κύρια στάδια:
\begin{enumerate}[label=\arabic*.]
    \item \textbf{Φόρτωση Δεδομένων:} Υποστήριξη για αρχεία μορφής AnnData (.h5ad).
    \item \textbf{Ενοποίηση Δεδομένων (Προαιρετικά):} Χρήση του αλγορίθμου Scanorama για τη διόρθωση batch effects όταν παρέχονται πολλαπλά δείγματα.
    \item \textbf{Ποιοτικός Έλεγχος (QC):} Φιλτράρισμα κυττάρων και γονιδίων βάσει παραμέτρων που ορίζει ο χρήστης (π.χ., min counts, min genes, mitochondrial percentage).
    \item \textbf{Ομαλοποίηση και Εντοπισμός Ιδιαίτερα Μεταβλητών Γονιδίων (HVGs):} Προετοιμασία των δεδομένων για downstream ανάλυση.
    \item \textbf{Μείωση Διαστατικότητας:} Ανάλυση Κύριων Συνιστωσών (PCA) και Uniform Manifold Approximation and Projection (UMAP).
    \item \textbf{Ομαδοποίηση Κυττάρων (Clustering):} Εφαρμογή του αλγορίθμου Leiden για τον εντοπισμό κυτταρικών συστάδων.
    \item \textbf{Ανάλυση Διαφορικά Εκφρασμένων Γονιδίων (DEG Analysis):} Εντοπισμός γονιδίων που εκφράζονται διαφορετικά μεταξύ των ορισμένων ομάδων.
    \item \textbf{Χαρακτηρισμός Κυτταρικών Τύπων (Cell Type Annotation):} Χρήση της βιβλιοθήκης Decoupler (μεθόδοι ORA και WMS) βάσει αρχείου δεικτών που παρέχει ο χρήστης.
    \item \textbf{Οπτικοποιήσεις:} Δημιουργία διαφόρων διαγραμμάτων (violin plots, UMAPs, volcano plots, heatmaps) για την εξερεύνηση των αποτελεσμάτων.
\end{enumerate}

Η εφαρμογή διαρθρώνεται γύρω από ένα κύριο αρχείο `app.py` που διαχειρίζεται τη διεπαφή χρήστη και τη ροή εργασιών, και μια σειρά από βοηθητικά scripts στον κατάλογο \texttt{pipeline\_scripts/} που περιέχουν τις συναρτήσεις για κάθε βήμα...

Η κατάσταση της εφαρμογής (session state) διατηρείται για να επιτρέπει τη διαδραστική ανάλυση χωρίς απώλεια των ενδιάμεσων αποτελεσμάτων.

% --- Μέχρι το σημείο των UML διαγραμμάτων δεν αλλάζεις τίποτα! ---

% --- UML Διαγράμματα ---
\section{UML Διαγράμματα}
\label{sec:uml}
Για την καλύτερη κατανόηση της αρχιτεκτονικής και της λειτουργικότητας της εφαρμογής, δημιουργήθηκαν τα παρακάτω UML διαγράμματα.

\subsection{Use Case Diagram (Διάγραμμα Περιπτώσεων Χρήσης)}
Το διάγραμμα περιπτώσεων χρήσης περιλαμβάνει τα εξής:
\begin{itemize}
    \item \textbf{Actor:} \textit{Ερευνητής / Χρήστης της Εφαρμογής}
    \item \textbf{Κύριες Περιπτώσεις Χρήσης:}
    \begin{itemize}
        \item \textbf{Φόρτωση δεδομένων} AnnData (.h5ad)
        \item \textbf{Εκτέλεση Ενοποίησης (Integration)} με Scanorama
        \item \textbf{Ποιοτικός Έλεγχος (QC)} και ορισμός παραμέτρων
        \item \textbf{Εκτέλεση Ομαλοποίησης, HVG selection, scaling}
        \item \textbf{Μείωση Διαστατικότητας (PCA, UMAP)}
        \item \textbf{Ομαδοποίηση (Clustering, Leiden)}
        \item \textbf{Ανάλυση DEGs (Differential Expression)}
        \item \textbf{Χαρακτηρισμός κυτταρικών τύπων (Cell Type Annotation, Decoupler)}
        \item \textbf{Οπτικοποίηση αποτελεσμάτων (UMAP, Violin και Volcano plots, Heatmaps)}
        \item \textbf{Λήψη/εξαγωγή των αποτελεσμάτων}
    \end{itemize}
\end{itemize}
\textit{(Δες Σχήμα~\ref{fig:use_case_diagram})}

\begin{figure}[H]
    \centering
    % Εδώ βάζεις export του PlantUML use case, ή αφήνεις το placeholder.
   \includegraphics[width=0.8\textwidth]{uml.png}
    \caption{Διάγραμμα Περιπτώσεων Χρήσης της εφαρμογής scRNA-seq.}
    \label{fig:use_case_diagram}

\end{figure}

\subsection{Class Diagram (Διάγραμμα Κλάσεων)}
Η εφαρμογή είναι αρθρωτή, με βασικές δομές και αρχεία:
\begin{itemize}
    \item \texttt{app.py} (\textbf{Κεντρική Εφαρμογή}): Ελέγχει το user interface, το session state, τη ροή pipeline και τις οπτικοποιήσεις.
    \item \texttt{pipeline\_scripts/}
    \begin{itemize}
        \item \texttt{preprocessing.py}: Ποιοτικός έλεγχος και ομαλοποίηση δεδομένων
        \item \texttt{integration.py}: Ενοποίηση με Scanorama
        \item \texttt{dimensionality\_reduction.py}: PCA/UMAP/Leiden clustering
        \item \texttt{deg\_analysis.py}: DEG analysis Και volcano plot functions
        \item \texttt{annotation.py}: Cell type annotation με decoupler (ORA/WMS)
    \end{itemize}
    \item \texttt{AnnData Object}: Κεντρικό data structure, μεταφέρεται και ενημερώνεται σε όλα τα βήματα (μέσα στο session state).
\end{itemize}

Οι συσχετίσεις συνοψίζονται:
\begin{itemize}
    \item Το \texttt{app.py} καλεί συναρτήσεις από τα υπόλοιπα scripts, ενημερώνοντας το AnnData και session state.
    \item Κάθε script λειτουργεί modular, επιστρέφοντας ενημερωμένο AnnData ή αποτελέσματα που χρησιμοποιούνται στο UI.
\end{itemize}


\newpage

% --- Ανάλυση της Υλοποίησης με Τεχνικές Λεπτομέρειες ---
\section{Ανάλυση της Υλοποίησης με Τεχνικές Λεπτομέρειες}
\label{sec:analysis_implementation}

\subsection{Κύρια Εφαρμογή (\texttt{app.py})}
Η εφαρμογή υλοποιεί διαδραστικό interface με χρήση Streamlit.
\begin{itemize}
    \item \textbf{UI Δομή:}
    \begin{itemize}
        \item Χρησιμοποιούνται \texttt{st.tabs} για διαχωρισμό των ενοτήτων (Ανάλυση Δεδομένων, Πληροφορίες Ομάδας).
        \item Όλα τα βήματα ρύθμισης γίνονται στο \texttt{st.sidebar}, όπου μέσω \texttt{st.expander} οργανώνονται οι παράμετροι κάθε pipeline βήματος (QC, Integration, Clustering κ.λπ.).
        \item Widgets όπως \texttt{st.file\_uploader}, \texttt{st.button}, \texttt{st.slider}, \texttt{st.selectbox}, \texttt{st.checkbox}, \texttt{st.text\_input}, \texttt{st.number\_input} χρησιμοποιούνται για παραμετροποίηση από τον χρήστη.
    \end{itemize}
    \item \textbf{Διαχείριση Session State:}
    \begin{itemize}
        \item Το \texttt{st.session\_state} χρησιμοποιείται εκτεταμένα για να διατηρείται η κατάσταση του pipeline ανά reload. Ενδεικτικές μεταβλητές: \texttt{adata\_initial\_list}, \texttt{adata\_processed}, \texttt{adata\_clustered}, \texttt{deg\_results}, \texttt{adata\_annotated}, \texttt{current\_stage}.
        \item Κάθε pipeline βήμα ανανεώνει τα κατάλληλα αντικείμενα (AnnData, αποτελέσματα DEG, annotations).
        \item Η χρήση του \texttt{st.rerun()} (π.χ. μετά από επιτυχή φόρτωση ή αλλαγή παραμέτρων) διασφαλίζει τον συγχρονισμό UI και δεδομένων.
    \end{itemize}
    \item \textbf{Workflow Logic:}
    \begin{itemize}
        \item Υπάρχει αυστηρή λογική ακολουθίας (pipeline): δεν μπορεί να προχωρήσει ο χρήστης χωρίς να ολοκληρώσει προηγούμενα βήματα (π.χ. δεν επιτρέπεται clustering αν δεν έχει γίνει QC).
        \item Κάθε ενέργεια (button press) οδηγεί σε εκτέλεση αντίστοιχης συνάρτησης στα pipeline scripts, με ελέγχους εγκυρότητας (π.χ. είναι τα δεδομένα φορτωμένα; QC completed; κλπ).
    \end{itemize}
    \item \textbf{Φόρτωση/Προεπισκόπηση Δεδομένων:}
    \begin{itemize}
        \item Φόρτωση πολλαπλών .h5ad γίνεται με \texttt{st.file\_uploader (accept\_multiple\_files=True)}, δημιουργώντας λίστα AnnData.
        \item Αν χρειάζεται, τα AnnData objects ενώνονται (concatenate) βάσει batch, ώστε να μπορεί να γίνει integration και joint analysis.
        \item Η εφαρμογή εμφανίζει πληροφορίες για το κάθε AnnData (shape, keys σε .obs/.var), δείγμα από τις μεταβλητές.
    \end{itemize}
    \item \textbf{Οπτικοποιήσεις:}
    \begin{itemize}
        \item Όλα τα plots δημιουργούνται είτε με scanpy/scanorama (π.χ. UMAP, violin, volcano, heatmap) και αποδίδονται στην εφαρμογή με \texttt{st.pyplot(fig)}.
        \item Ο χρήστης μπορεί να διαλέξει τι θα οπτικοποιήσει (UMAP by cluster, by batch, by gene expression κλπ).
    \end{itemize}
\end{itemize}

\subsection{Scripts του Pipeline (\texttt{pipeline\_scripts/})}
Ακολουθεί ανάλυση κάθε αρχείου:

\begin{itemize}
    \item \textbf{preprocessing.py:} Περιλαμβάνει τη \texttt{preprocess\_adata}. Κάνει:
        \begin{itemize}
            \item Υπολογισμό QC metrics με \texttt{sc.pp.calculate\_qc\_metrics} (π.χ. n\_genes\_by\_counts, total\_counts, pct\_counts\_mt)
            \item Φιλτράρισμα κυττάρων/γονιδίων με βάση τα thresholds του χρήστη
            \item Ομαλοποίηση counts (\texttt{sc.pp.normalize\_total}), log1p transform (\texttt{sc.pp.log1p}), επιλογή HVGs (\texttt{sc.pp.highly\_variable\_genes}), scaling (\texttt{sc.pp.scale})
            \item Διατηρείται το αρχικό raw matrix στο \texttt{adata.raw} πριν την επιλογή HVG
        \end{itemize}
    \item \textbf{integration.py:} Η \texttt{run\_scanorama\_integration} παίρνει λίστα AnnData και ενώνει τα δείγματα με Scanorama.
        \begin{itemize}
            \item Κάνει join/merge AnnData βάσει \texttt{batch\_key}, αφαιρεί batch effect.
            \item Αποθηκεύει το embedding στο \texttt{adata.obsm['X\_scanorama']}
        \end{itemize}
    \item \textbf{dimensionality\_reduction.py:} Η \texttt{run\_pca\_umap\_leiden} τρέχει PCA, υπολογίζει neighbors, UMAP και Leiden clustering.
        \begin{itemize}
            \item \texttt{sc.tl.pca} υπολογίζει τις κύριες συνιστώσες.
            \item \texttt{sc.pp.neighbors} φτιάχνει neighborhood graph (παράμετροι: n\_neighbors, n\_pcs)
            \item \texttt{sc.tl.umap} για 2D embedding (min\_dist, spread)
            \item \texttt{sc.tl.leiden} για clustering (resolution)
            \item Αποθήκευση αποτελεσμάτων σε \texttt{adata.obsm['X\_pca']}, \texttt{adata.obsm['X\_umap']}, \texttt{adata.obs['leiden']}
        \end{itemize}
    \item \textbf{deg\_analysis.py:} Η \texttt{run\_deg\_analysis} καλεί \texttt{sc.tl.rank\_genes\_groups} με method ('wilcoxon', 't-test', 'logreg'). Παράγει dataframe με logFC, p-values, adj p-values για όλα τα γονίδια/συγκρίσεις.
        \begin{itemize}
            \item Επιλογή ομάδας αναφοράς και συγκρινόμενης ομάδας (groupby\_key, group1, reference)
            \item \texttt{plot\_volcano} (στο app.py) φτιάχνει volcano plot για επιλεγμένη σύγκριση (logFC vs -log10(padj))
        \end{itemize}
    \item \textbf{annotation.py:} Περιέχει τις \texttt{create\_marker\_dict\_from\_df}, \texttt{run\_decoupler\_ora}, \texttt{run\_decoupler\_wmean}.
        \begin{itemize}
            \item \texttt{create\_marker\_dict\_from\_df} μετατρέπει ένα CSV με markers σε λεξικό (cell type -> gene list)
            \item \texttt{run\_decoupler\_ora}, \texttt{run\_decoupler\_wmean} τρέχουν enrichment analysis με decoupler (ORA, WMS) και αποθηκεύουν τα scores σε \texttt{adata.obsm['ora\_estimate']}, \texttt{adata.obs['wms\_scores']}
        \end{itemize}
\end{itemize}

\subsection{Διαχείριση Δεδομένων AnnData}
Η ροή του AnnData object:
\begin{itemize}
    \item \textbf{Αρχική Φόρτωση:} \texttt{.X} = raw counts, \texttt{.obs} = μεταδεδομένα για cells, \texttt{.var} = πληροφορίες γονιδίων
    \item \textbf{Μετά το QC/Normalization/HVG:}
        \begin{itemize}
            \item \texttt{.X} ενημερώνεται με normalized/log1p counts
            \item \texttt{.obs} προστίθενται QC fields (\texttt{n\_genes\_by\_counts}, \texttt{total\_counts}, \texttt{pct\_counts\_mt})
            \item \texttt{.var} προστίθενται (\texttt{highly\_variable}, \texttt{dispersions\_norm})
            \item \texttt{.raw} κρατάει immutable version για downstream DEG
        \end{itemize}
    \item \textbf{PCA/UMAP/Leiden:}
        \begin{itemize}
            \item \texttt{.obsm['X\_pca']} = coordinates PCA, \texttt{.obsm['X\_umap']} = UMAP embedding
            \item \texttt{.obs['leiden']} = cluster assignments
        \end{itemize}
    \item \textbf{DEG ανάλυση:}
        \begin{itemize}
            \item Αποτελέσματα στο \texttt{.uns['rank\_genes\_groups']}
        \end{itemize}
    \item \textbf{Annotation:}
        \begin{itemize}
            \item ORA scores στο \texttt{.obsm['ora\_estimate']}
            \item WMS scores ως νέα στήλη στο \texttt{.obs}
        \end{itemize}
\end{itemize}

\subsection{Προκλήσεις και Λύσεις}
\begin{itemize}
    \item \textbf{Διαχείριση Μεγάλων Δεδομένων:} Η ανάλυση scRNA-seq οδηγεί συχνά σε τεράστια matrices. Χρησιμοποιήθηκε φειδωλή χρήση \texttt{.copy()} για αποφυγή memory spikes. Επίσης, γίνεται επιλογή HVGs για μείωση διαστάσεων και πειραματισμός με μικρότερα datasets στη φάση ανάπτυξης.
    \item \textbf{Session State στο Streamlit:} Η διατήρηση ενδιάμεσης κατάστασης κρίσιμη για interactive ανάλυση. Αντιμετωπίστηκαν προβλήματα με το auto-rerun και χρησιμοποιήθηκαν buttons και λογική για να αποφεύγονται ανεπιθύμητα refresh.
    \item \textbf{Συμβατότητα Εξαρτήσεων:} Πολλά πακέτα (π.χ. scanpy, hdf5plugin) είναι δύσκολο να εγκατασταθούν σε Windows. Εφαρμόστηκε χρήση WSL, και τελικά Dockerization με \texttt{python:3.12-slim} image, για πλήρη reproducibility.
    \item \textbf{Διαδραστικότητα Οπτικοποιήσεων:} Οι οπτικοποιήσεις (π.χ. UMAP, Volcano) πρέπει να ενημερώνονται άμεσα με βάση τις επιλογές του χρήστη. Εφαρμόστηκε ανανέωση plots μέσω button triggers και προσεκτικός χειρισμός session state.
\end{itemize}

% --- Οπτικοποιήσεις των Αποτελεσμάτων που Παράγει η Εφαρμογή ---
\section{Οπτικοποιήσεις των Αποτελεσμάτων που Παράγει η Εφαρμογή}
\label{sec:visualizations}
Η εφαρμογή παράγει:
\begin{itemize}
    \item \textbf{Violin Plots} για QC (πριν/μετά) σε metrics: αριθμός γονιδίων/cell, συνολικό count, %mitochondrial
    \item \textbf{UMAP Plots} με δυνατότητα χρωματισμού: clusters, batch, condition, gene expression
    \item \textbf{Volcano Plots} για τα DEGs (logFC vs -log10(adj. p-value))
    \item \textbf{Heatmaps} για ORA scores και \textbf{Violin Plots} για WMS (cell type annotation)
\end{itemize}

% (τα placeholders των εικόνων τα κρατάς ή προσθέτεις τις δικές σου εικόνες)

% --- Περιγραφή Dockerization της Εφαρμογής ---
\section{Περιγραφή Dockerization της Εφαρμογής}
\label{sec:dockerization}
\subsection{Dockerfile}
\begin{itemize}
    \item \texttt{FROM python:3.12-slim}: Λεπτή (slim) εικόνα για μικρό μέγεθος, χωρίς περιττά packages.
    \item \texttt{WORKDIR /app}: Όλες οι εντολές εκτελούνται στο /app.
    \item \texttt{COPY requirements.txt .} και \texttt{RUN pip install ...}: Πρώτα εγκαθίστανται οι εξαρτήσεις, ώστε να γίνεται caching και να μη χτίζεται ξανά όλο το image σε μικρές αλλαγές στον κώδικα.
    \item \texttt{COPY app.py .} και \texttt{COPY pipeline\_scripts/ ./pipeline\_scripts/}: Κώδικας εφαρμογής.
    \item \texttt{EXPOSE 8501}: Ανοίγει το port για το Streamlit server.
    \item \texttt{ENV ...}: Περιβάλλον για Streamlit (π.χ. headless, port)
    \item \texttt{HEALTHCHECK ...}: Ελέγχει εάν τρέχει το app στο port 8501 (endpoint health)
    \item \texttt{CMD ["streamlit", "run", "app.py"]}: Εκκινεί το Streamlit με το app.
\end{itemize}

(Παράδειγμα Dockerfile παρατίθεται στο listings)

\subsection{.dockerignore}
Στο .dockerignore εξαιρούνται:
\begin{itemize}
    \item \texttt{.git}, \texttt{.venv}, \texttt{\_\_pycache\_\_}, \texttt{data/} -- για μείωση μεγέθους image και αποφυγή sensitive data
    \item Η χρήση external volume (\texttt{-v ...}) ενθαρρύνεται για τα δεδομένα.
\end{itemize}

\subsection{Build/Run}
\begin{itemize}
    \item \texttt{docker build -t streamlit-scrna-app .} το \texttt{-t} δίνει όνομα, η τελεία δηλώνει το path του Dockerfile.
    \item \texttt{docker run -p 8501:8501 -v \$\{PWD\}/data/h5ad\_filt:/app/data/h5ad\_filt streamlit-scrna-app} τρέχει το container, αντιστοιχίζει το local port 8501 και κάνει mount τα αρχεία δεδομένων.
    \item Σημαντικό: Τα δεδομένα πρέπει να βρίσκονται σε \texttt{data/h5ad\_filt} τοπικά, ώστε να είναι προσβάσιμα.
\end{itemize}

% --- Συμπεράσματα ---
\section{Συμπεράσματα}
\label{sec:symperasmata}
Η εργασία ολοκληρώθηκε με επιτυχία:
\begin{itemize}
    \item Δημιουργήθηκε πλήρως παραμετροποιήσιμη και επεκτάσιμη web εφαρμογή για ανάλυση scRNA-seq με Streamlit.
    \item Όλο το pipeline (QC, integration, clustering, DEG, annotation) γίνεται από ένα φιλικό interface, με ενδιάμεση αποθήκευση και εύκολες οπτικοποιήσεις.
    \item Υιοθετήθηκε Dockerization για αποφυγή προβλημάτων εξαρτήσεων και 100\% reproducibility.
    \item Αντιμετωπίστηκαν δυσκολίες (μεγάλα δεδομένα, streamlit state, πακέτα) με έξυπνες τεχνικές.
    \item Προοπτικές: Υποστήριξη advanced ανάλυσης (trajectories), διαδραστικά/3D plots, προσθήκη annotation databases.
\end{itemize}

% --- Βιβλιογραφία ---
\section*{Βιβλιογραφία}
\addcontentsline{toc}{section}{Βιβλιογραφία}
\begin{thebibliography}{99}
    \bibitem{scanpy} Wolf, F. A., Angerer, P., \& Theis, F. J. (2018). SCANPY: large-scale single-cell gene expression data analysis. Genome biology, 19(1), 1-5.
    \bibitem{streamlit} Streamlit Documentation. (2024). \url{https://docs.streamlit.io}
    \bibitem{decoupler} Badia-i-Mompel, P., et al. (2022). decoupler: Ensemble methods for gene set analysis. Bioinformatics, 38(7), 1929–1931.
    \bibitem{scanorama} Hie, B., et al. (2019). Efficient integration of heterogeneous single-cell transcriptomes using Scanorama. Nature Biotechnology, 37(6), 685–691.
    \bibitem{leidenalg} Traag, V. A., Waltman, L., and van Eck, N. J. (2019). From Louvain to Leiden: guaranteeing well-connected communities. Scientific Reports, 9, 5233.
\end{thebibliography}
\end{document}